\documentclass[letter,10pt]{article}
\usepackage{TLCresume}
\begin{document}

%====================
% EXPERIENCE A
%====================
\subsection{{Researcher - Data Science  \hfill September 2017 - Current}}
\subtext{ATLAS Experiment | CERN \hfill Toronto, Canada \\ \small{ATLAS is a general-purpose particle physics experiment at the Large Hadron Collider at CERN designed to answer questions about the fundamental forces of nature. As of 2022, it has over 5000 collaborating members from 181 institutions in 42 countries.}}
\begin{zitemize}
\item Led efforts in a team of 10 to improve data quality and increased the statistical significance from 99.4\% to over 99.9999999\% (4$\sigma$ to 6.5$\sigma$) for a signal model of less than $e^{-13}$ effect 
%item Analysis goal was to reject the background only likelihood model at 99.99994\% confidence level, and to set limits on the strength of alternative signal likelihood models.
%\item Calibrated data objects to correct for detector biases and errors introduced in the data-taking process, 
%\item Built the internal C++ analysis software used by 10+ team members to correct for mis-modelling in physics simulation with data-driven methods \\
\item Manipulated 1000+ TB datasets use over the CERN computing grid (similar to AWS), filtering down to a 2 GB final dataset for causal inference, which was done with a statistical model of 500+ sources of error all constrained with data\\
\item Performed A/B testing comparing deep neural networks built with TensorFlow, decision trees built with XGBoost, 
and a strong baseline of physics logic filters, taking into account signal significance related KPIs and business logic (physics), increasing signal acceptance by a factor of 3 
%\item Collaborated closely with 5 other project teams to ensure compatibility in analysis methods and prepared for combined interpretations across multiple projects
%\item Hosted regular communications with internal stakeholders and clients to iterate analysis scopes and goals
%\item Personally developed the software framework to apply cuts to datasets, took over dataset production over distributed computing tools, and lead the development of the statistical tools
%\item Developed the probability model used to perform hypothesis testing 
\item Wrote and edited over 500 pages of publication ready presentations and documents, some of which have 100+ citations, and presented at conferences both internal and external
%\item Presented findings to general audience in various outreach events
%\item Addressed different requirements from 100+ collaborating institutions and iterated analysis methods to satisfy 2000+ scientific authors whose names will appear on the final paper
%\item Communicated research results and findings to non-specialists in outreach talks and events
%\item Authored on all ATLAS publications since 2019, over 5000 (and counting)
\end{zitemize}
%====================
% EXPERIENCE A
%====================
\subsection{{Developer - Detector Simulation  \hfill August 2017 - September 2021 }}
\subtext{ATLAS Experiment | CERN \hfill Toronto, Canada }
\begin{zitemize}
\item Contributed to an open-source physics engine (\href{https://github.com/acts-project/acts/}{acts}) to simulate detector response to charged particles
\item Optimized physics simulation time with a library of pre-processed detector responses as a function of input parameters using C++ and python, for a simulation engine used by 2000+ scientists 
\item Improved simulation time by a factor of 10, from 28.1s/event to 2.82s/event
\end{zitemize}

%====================
% EXPERIENCE B
%====================

\subsection{{Teaching Assistant  \hfill September 2016 – September 2021}}
\subtext{University of Toronto \hfill Toronto, Canada}
\begin{zitemize}
\item Ran weekly tutorials, hosted office hours, and graded exams for undergraduate physics courses of up to 150 students
\item Communicated topics include mechanics, electromagnetism, quantum mechanics, and relativity to non-specialists
\end{zitemize}
%====================
% EXPERIENCE E
%====================
%\subsection{{ROLE / PROJECT E \hfill MMM YYYY --- MMM YYYY}}
%\subtext{company E \hfill somewhere, state}
%\begin{zitemize}
%\item In lobortis libero consectetur eros vehicula, vel pellentesque quam fringilla.
%\item Ut malesuada purus at mi placerat dapibus.
%\item Suspendisse finibus massa eu nisi dictum, a imperdiet tellus convallis.
%\item Nam feugiat erat vestibulum lacus feugiat, efficitur gravida nunc imperdiet.
%\item Morbi porta lacus vitae augue luctus, a rhoncus est sagittis.
%\end{zitemize}

\end{document}