\documentclass[letter,10pt]{article}
\usepackage{TLCresume}
\begin{document}

%====================
% EXPERIENCE A
%====================
\subsection{{Researcher - Data Analysis  \hfill September 2017 - 2022}}
\subtext{ATLAS Experiment | CERN \hfill Toronto, Canada \\ \small{ATLAS is a general-purpose particle physics experiment at the Large Hadron Collider at CERN designed to answer questions about the fundamental forces of nature. As of 2022, it has over 5000 collaborating members from 181 institutions in 42 countries.}}
\begin{zitemize}
\item Led efforts in a team of 10 to improve data quality and increased the expected statistical significance from 99.4\% to over 99.9999999\% (4$\sigma$ to 6.5$\sigma$), vastly outperforming the goal of 5$\sigma$ set by the collaboration
%item Analysis goal was to reject the background only likelihood model at 99.99994\% confidence level, and to set limits on the strength of alternative signal likelihood models.
%\item Calibrated data objects to correct for detector biases and errors introduced in the data-taking process, 
\item Corrected for mis-modelling in physics simulation with data-driven methods, and calibrated data objects to correct for biases and errors in the detector \\
Software used include C++, python, Git, and ROOT (similar to R)
\item Skim data sets for final analysis use over distributed computing resources, from  1000+ TB to less than 10 GB \\
Software used include the LHC computing grid (similar to AWS) and rucio (a scientific data management tool)
\item Optimized deep neural nets with feature engineering and recursive feature elimination, increasing signal acceptance by a factor of 3 from 543.05 events to 1713.2 events
\item Led a team of 5 scientists in the development of a probability model with 3 observed data sets and 100+ nuisance parameters
\item Collaborated closely with other project teams to ensure compatibility in analysis methods and prepared for combined statistical interpretations across multiple projects
%\item Personally developed the software framework to apply cuts to datasets, took over dataset production over distributed computing tools, and lead the development of the statistical tools
%\item Personally debugged the probability model, and resolved a major conflict in the probability model by identifying mis-modelling in NN inputs due to data calibration, and moved the team towards publication
\item Documented and edited over 500 pages of notes for internal use, and presented monthly status updates to lead physicists using a variety of data visualization tools
\item Addressed different requirements from all collaborating institutions and iterated analysis methods to satisfy all 3000 scientific authors whose names will appear on the final paper
\item Communicated research results and findings to non-specialists in outreach talks and events
%\item Authored on all ATLAS publications since 2019, over 5000 (and counting)
\end{zitemize}
%====================
% EXPERIENCE A
%====================
\subsection{{Developer - Detector Simulation  \hfill August 2017 - September 2021 }}
\subtext{ATLAS Experiment | CERN \hfill Toronto, Canada }
\begin{zitemize}
\item Used a multi-thread physics engine (\href{https://github.com/acts-project/acts/}{acts}) to simulate detector response to charged particles
\item Optimized physics simulation time with a library of pre-processed detector responses as a function of input parameters \\
Software used include C++, python, Docker, Valgrind, and Git CI/CD
\item Improved simulation time by a factor of 10, from 28.1s/event to 2.82s/event
\end{zitemize}

%====================
% EXPERIENCE B
%====================

\subsection{{Teaching Assistant  \hfill September 2016 – September 2021}}
\subtext{University of Toronto \hfill Toronto, Canada}
\begin{zitemize}
\item Ran weekly tutorials, hosted office hours, and graded exams for undergraduate physics courses of up to 150 students
\item Topics taught include mechanics, electromagnetism, quantum mechanics, and relativity
\end{zitemize}
%====================
% EXPERIENCE E
%====================
%\subsection{{ROLE / PROJECT E \hfill MMM YYYY --- MMM YYYY}}
%\subtext{company E \hfill somewhere, state}
%\begin{zitemize}
%\item In lobortis libero consectetur eros vehicula, vel pellentesque quam fringilla.
%\item Ut malesuada purus at mi placerat dapibus.
%\item Suspendisse finibus massa eu nisi dictum, a imperdiet tellus convallis.
%\item Nam feugiat erat vestibulum lacus feugiat, efficitur gravida nunc imperdiet.
%\item Morbi porta lacus vitae augue luctus, a rhoncus est sagittis.
%\end{zitemize}

\end{document}