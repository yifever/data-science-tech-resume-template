\documentclass[letter,10pt]{article}
\usepackage{TLCresume}
\begin{document}

%====================
% EXPERIENCE A
%====================
\subsection{{Researcher - Data Analysis  \hfill Sept 2017 --- }}
\subtext{ATLAS Experiment | CERN \hfill Toronto, Canada \\ \small{ATLAS is a general-purpose particle physics experiment at the Large Hadron Collider at CERN designed to answer questions about the fundamental forces of nature. As of 2022, it has over 5000 collaborating members from 181 institutions in 42 countries.}}
\begin{zitemize}
\item Collaborate in a large experiment with over 5000 members and 3000 authored scientists 
\item Apply cuts to data sets according to the underlying physics to define phase spaces where the desired signal is enhanced
\item Calibrate different data objects with both traditional and machine-learning based approaches to correct for biases and errors introduced in the data-taking process
\item Skim data sets for final analysis use over distributed computing resources, from  1000+ TB to less than 10 GB
\item Correct for various mis-modelling in physics simulation with data-driven methods like ABCD extrapolation
\item Train and use recursive neural networks as the discriminant for signal separation 
\item Perform likelihood maximization on a probability model with 3 observed data sets and 100+ nuisance parameters
%\item Authored on all ATLAS publications since 2019, over 5000 (and counting)
\end{zitemize}
%====================
% EXPERIENCE A
%====================
\subsection{{Developer - Detector Simulation  \hfill Aug 2017 -- Sept 2021 }}
\subtext{ATLAS Experiment | CERN \hfill Toronto, Canada }
\begin{zitemize}
\item Use multi-thread physics engines to simulate detector response to charged particles
\item Build interface between in-house software framework and third-party physics libraries
\item Optimize physics simulation time with a library of pre-processed detector responses as a function of input parameters
\item Validate physics simulation accuracy for each detector component by comparing a high- and low-level output variables
\end{zitemize}

%====================
% EXPERIENCE B
%====================

\subsection{{Teaching Assistant  \hfill Sept 2016 –- Sept 2021}}
\subtext{University of Toronto \hfill Toronto, Canada}
\begin{zitemize}
\item Run weekly tutorials and host office hours for undergraduate physics courses
\item Grade lab reports, essays, assignments and exams, for classes of up to 150 students
\item Topics taught include mechanics, electromagnetism, quantum mechanics, and relativity
\end{zitemize}
%====================
% EXPERIENCE E
%====================
%\subsection{{ROLE / PROJECT E \hfill MMM YYYY --- MMM YYYY}}
%\subtext{company E \hfill somewhere, state}
%\begin{zitemize}
%\item In lobortis libero consectetur eros vehicula, vel pellentesque quam fringilla.
%\item Ut malesuada purus at mi placerat dapibus.
%\item Suspendisse finibus massa eu nisi dictum, a imperdiet tellus convallis.
%\item Nam feugiat erat vestibulum lacus feugiat, efficitur gravida nunc imperdiet.
%\item Morbi porta lacus vitae augue luctus, a rhoncus est sagittis.
%\end{zitemize}

\end{document}